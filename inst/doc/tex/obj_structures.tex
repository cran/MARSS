\chapter{Package MARSS:  Object structures}

\section{Model objects: class marssm}
Objects of class `marssm'\index{objects!marssm} specify Multivariate Autoregressive State Space (MARSS) models. The \verb@model@ component of an ML estimation object (class marssMLE; see below) is a marssm object. These objects have the following components:

  \begin{description}
    \item[\textbf{data}]{ An optional matrix (not dataframe), in which each row is a time series (time across columns). }
    \item[\textbf{fixed}]{ A list with 8 matrices Z, A, R, B, U, Q, x0, V0, specifying which elements of each parameter are fixed. }
    \item[\textbf{free}]{ A list with 8 matrices Z, A, R, B, U, Q, x0, V0, specifying which elements of each parameter are to be estimated. }
    \item[\textbf{M}]{ An array of dim $n \times n \times T$ (an $n \times n$ missing values matrix for each time point).  Each matrix is diagonal with 0 at the $i,i$ value if the $i$-th value of $\yy$ is missing, and 1 otherwise.} 
    \item[\textbf{miss.value}]{ Specifies missing value representation in the data. }
  \end{description}

  The matrices in \verb@fixed@ and \verb@free@ work as pairs to specify the fixed and free elements for each parameter. See Chapter \ref{chap:modelspec}. The dimensions for \verb@fixed@ and \verb@free@ matrices are as follows, where $n$ is the number of observation time series and $m$ is the number of state processes:
  \begin{description}
    \item[\textbf{Z}]{ n x m }
    \item[\textbf{B}]{ m x m }
    \item[\textbf{U}]{ m x 1 }
    \item[\textbf{Q}]{ m x m }
    \item[\textbf{A}]{ n x 1 }
    \item[\textbf{R}]{ n x n }
    \item[\textbf{x0}]{ m x 1 }
    \item[\textbf{V0}]{ m x m }
  \end{description} 

Use \verb@is.marssm()@\index{functions!is.marssm} to check whether an marssm object is correctly specified. The MARSS package includes an \verb@as.marssm()@\index{functions!as.marssm} method to convert objects of class popWrap (see next section) to objects of class marssm. 

\section{Wrapper objects: class popWrap}\index{objects!popWrap}

Wrapper objects of class popWrap\index{objects!popWrap} contain specifications and options for estimation of a MARSS model. A popWrap object has the following components:

  \begin{description}
  
  \item[\textbf{data}]{ A matrix (not dataframe) of observations (rows) $\times$ time (columns).  }
  \item[\textbf{m}]{ Number of hidden state processes (number of rows in $\xx$). }
  \item[\textbf{constraint}]{ Either a list with 8 string elements Z, A, R, B, U, Q, x0, V0 (see below for details), or string \verb@"use fixed/free"@. }
  \item[\textbf{fixed}]{ If \verb@constraint[[elem]]="use fixed/free"@, a list with 8 matrices Z, A, R, B, U, Q, x0, V0. }
  \item[\textbf{free}]{ If \verb@constraint[[elem]]="use fixed/free"@, a list with 8 matrices Z, A, R, B, U, Q, x0, V0. }
  \item[\textbf{inits}]{ A list with 8 matrices Z, A, R, B, U, Q, x0, V0, specifying initial values for parameters. Dimensions are given in the class `marssm' section. }
  \item[\textbf{miss.value}]{ Specifies missing value representation (default is -99). }
  \item[\textbf{method}]{ The method used for estimation: `kem' for Kalman-EM, `BFGS' for quasi-Newton.}
  \item[\textbf{control}]{ List of estimation options.  For the EM algorithm, these include the elements 
    \verb@minit, maxit, abstol, iter.V0, safe@ and \verb@trace@. For Monte Carlo initialization, these include the elements \verb@MCInit@, \verb@numInits@, \verb@numInitSteps@ and \verb@boundsInits@. See class marssMLE section for details. }  
  \end{description}

Component \verb@constraint@ is a convenient way to specify model structure for certain common cases.  If \verb@constraint="use fixed/free"@, both \verb@fixed@ and \verb@free@ must be provided.  See the class marssm section for how to specify fixed and free matrices.  The function \verb@MARSS()@\index{functions!MARSS} calls \verb@popWrap()@ to create a popWrap object, then \verb@is.marssm()@ to coerce this object to class marssm for the estimation function.

  The \verb@popWrap()@\index{functions!popWrap} function calls \verb@checkPopWrap()@\index{functions!checkPopWrap} to check user inputs from \verb@MARSS()@. Valid constraints are below.  

  \begin{description}   
    \item[\textbf{A}]{ May be either the string `scaling' or the string `zero' to specify a column vector of zeros ($\aa=0$).}
    \item[\textbf{B}]{ String `identity' or a numeric matrix specifying a fixed $\BB$ matrix.  The string `zero' may be used to specify a $m \times m$ matrix of zeros ($\BB=0$).}
    \item[\textbf{Q}]{ String `unconstrained', `diagonal and unequal', `diagonal and equal', or `equalvarcov'. May also be numeric or character vector  of class factor specifying shared diagonal values or a numeric matrix specifying a fixed $\QQ$ matrix. } 
    \item[\textbf{R}]{ String `unconstrained', `diagonal and unequal', `diagonal and equal', or `equalvarcov'. May also be numeric or character vector  of class factor specifying shared diagonal values or a numeric matrix specifying a fixed $\RR$ matrix. }
    \item[\textbf{U}]{ String `unconstrained'=`unequal', or `equal'. May also be numeric or character vector  of class factor specifying shared $\uu$ elements or a $m \times 1$ numeric matrix specifying a fixed $\uu$ matrix. The string `zero' may be used to specify a column vector of zeros ($\uu=0$).}
    \item[\textbf{x0}]{ String `unconstrained'=`unequal', or `equal'. May also be vector  of class factor specifying shared $\pipi$ ($t=1$) values or a $m \times 1$ numeric matrix specifying a fixed $\pipi$ ($t=1$) matrix. The string `zero' may be used to specify a column vector of zeros ($\pipi=0$).}
    \item[\textbf{Z}]{ A vector  of class factor specifying which $\yy$ time series correspond to which state time series (in $\xx$) or a numeric $n \times m$ matrix specifying the $\ZZ$ matrix. The string `identity' can be used to specify a $n \times n$ identity matrix and string `ones' may be used to specify a column vector of $n$ ones.}
  \end{description}
  

\section{ML estimation objects: class marssMLE}\index{objects!marssMLE}

Objects of class marssMLE\index{objects!marssMLE} specify maximum-likelihood estimation for a MARSS model, both before and after fitting. A minimal marssMLE object contains components \verb@model, start@ and \verb@control@, which must be present for estimation by functions like \verb@MARSSkem()@\index{functions!MARSSkem}.

  \begin{description}
    \item[\textbf{model}]{ MARSS model specification (an object of class `marssm'). }
    \item[\textbf{start}]{ List with 7 matrices A, R, B, U, Q, x0, V0, specifying initial values for parameters. Dimensions are given in the class marssm section. }
    \item[\textbf{control}]{ A list specifying estimation options. For \verb@method="kem"@, these are
    \begin{description}
      \item[\textit{minit}]{ Minimum number of iterations in the maximization algorithm. } 
      \item[\textit{maxit}]{ Maximum number of iterations in the maximization algorithm. } 
      \item[\textit{abstol}]{ Optional tolerance for log-likelihood change.  If log-likelihood decreases less than this amount relative to the previous iteration, the EM algorithm exits. } 
      \item[\textit{iter.V0}]{ Maximum number of iterations for final likelihood calculation with V0 = 0. }
      \item[\textit{trace}]{ A positive integer. If not zero, a record will be created of each variable the maximization iterations. The information recorded depends on the maximization method.}
      \item[\textit{safe}]{If TRUE, \verb@MARSSkem()@ will rerun \verb@MARSSkf()@ after each individual parameter update rather than only after all parameters are updated.  }
      \item[\textit{MCInit}]{ Use Monte Carlo initialization? } 
      \item[\textit{numInits}]{ Number of random initial value draws. } 
      \item[\textit{numInitSteps}]{ Number of iterations for each initial value draw. }
      \item[\textit{boundsInits}]{ Bounds on the uniform distributions from which initial values will be drawn. (Note that bounds for the covariance matrices Q and R, which require positive values, are specified in logs.) }
      \item{\textit{silent}}{ Suppresses printing of progress bar and convergence information. }    
    \end{description}
  }
  \end{description}

\verb@MARSSkem()@\index{functions!MARSSkem} appends the following components to the marssMLE' object: 

  \begin{description}
  \item[\textbf{method}]{ A string specifying the estimation method (`kem' for estimation by \verb@MARSSkem()@). }
  \item[\textbf{par}]{ A list with 8 matrices of estimated parameter values Z, A, R, B, U, Q, x0, V0. If there are fixed elements in the matrices, the corresponding elements in \verb@$par@ are set to the fixed values.}
  \item[\textbf{kf}]{ A list containing Kalman filter/smoother output. See Chapter \ref{chap:mainfunctions} }
  \item[\textbf{numIter}]{ Number of iterations required for convergence. }
  \item[\textbf{convergence}]{ Convergence status. }
  \item[\textbf{logLik}]{ the exact Log-likelihood. See Section \ref{sec:exactlikelihood}.}
  \item[\textbf{errors}]{ any error messages }
  \item[\textbf{iter.record}]{ record of the parameter values at each iteration (if \verb@control$trace=1@) }
\end{description}

Several functions append additional components to the `marssMLE' object\index{objects!marssMLE} passed in. These include:

  \begin{description}
  \item{\verb@MARSSaic@}{ Appends \verb@AIC, AICc, AICbb, AICbp@, depending on the AIC flavors requested.\index{functions!MARSSaic} }
  \item{\verb@MARSShessian@}{ Appends \verb@Hessian, gradient, parMean@ and \verb@parSigma@.\index{functions!MARSShessian} }
  \item{\verb@MARSSparamCIs@}{ Appends \verb@par.se, par.bias, par.upCI@ and \verb@par.lowCI@.\index{functions!MARSSparamCIs}}
  \end{description}

